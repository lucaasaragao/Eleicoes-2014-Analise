\documentclass[]{article}
\usepackage{lmodern}
\usepackage{amssymb,amsmath}
\usepackage{ifxetex,ifluatex}
\usepackage{fixltx2e} % provides \textsubscript
\ifnum 0\ifxetex 1\fi\ifluatex 1\fi=0 % if pdftex
  \usepackage[T1]{fontenc}
  \usepackage[utf8]{inputenc}
\else % if luatex or xelatex
  \ifxetex
    \usepackage{mathspec}
  \else
    \usepackage{fontspec}
  \fi
  \defaultfontfeatures{Ligatures=TeX,Scale=MatchLowercase}
\fi
% use upquote if available, for straight quotes in verbatim environments
\IfFileExists{upquote.sty}{\usepackage{upquote}}{}
% use microtype if available
\IfFileExists{microtype.sty}{%
\usepackage{microtype}
\UseMicrotypeSet[protrusion]{basicmath} % disable protrusion for tt fonts
}{}
\usepackage[margin=1in]{geometry}
\usepackage{hyperref}
\hypersetup{unicode=true,
            pdftitle={Relação entre votos, receitas, despesas e bens de candidatos},
            pdfauthor={Lucas Aragão},
            pdfborder={0 0 0},
            breaklinks=true}
\urlstyle{same}  % don't use monospace font for urls
\usepackage{color}
\usepackage{fancyvrb}
\newcommand{\VerbBar}{|}
\newcommand{\VERB}{\Verb[commandchars=\\\{\}]}
\DefineVerbatimEnvironment{Highlighting}{Verbatim}{commandchars=\\\{\}}
% Add ',fontsize=\small' for more characters per line
\usepackage{framed}
\definecolor{shadecolor}{RGB}{248,248,248}
\newenvironment{Shaded}{\begin{snugshade}}{\end{snugshade}}
\newcommand{\KeywordTok}[1]{\textcolor[rgb]{0.13,0.29,0.53}{\textbf{#1}}}
\newcommand{\DataTypeTok}[1]{\textcolor[rgb]{0.13,0.29,0.53}{#1}}
\newcommand{\DecValTok}[1]{\textcolor[rgb]{0.00,0.00,0.81}{#1}}
\newcommand{\BaseNTok}[1]{\textcolor[rgb]{0.00,0.00,0.81}{#1}}
\newcommand{\FloatTok}[1]{\textcolor[rgb]{0.00,0.00,0.81}{#1}}
\newcommand{\ConstantTok}[1]{\textcolor[rgb]{0.00,0.00,0.00}{#1}}
\newcommand{\CharTok}[1]{\textcolor[rgb]{0.31,0.60,0.02}{#1}}
\newcommand{\SpecialCharTok}[1]{\textcolor[rgb]{0.00,0.00,0.00}{#1}}
\newcommand{\StringTok}[1]{\textcolor[rgb]{0.31,0.60,0.02}{#1}}
\newcommand{\VerbatimStringTok}[1]{\textcolor[rgb]{0.31,0.60,0.02}{#1}}
\newcommand{\SpecialStringTok}[1]{\textcolor[rgb]{0.31,0.60,0.02}{#1}}
\newcommand{\ImportTok}[1]{#1}
\newcommand{\CommentTok}[1]{\textcolor[rgb]{0.56,0.35,0.01}{\textit{#1}}}
\newcommand{\DocumentationTok}[1]{\textcolor[rgb]{0.56,0.35,0.01}{\textbf{\textit{#1}}}}
\newcommand{\AnnotationTok}[1]{\textcolor[rgb]{0.56,0.35,0.01}{\textbf{\textit{#1}}}}
\newcommand{\CommentVarTok}[1]{\textcolor[rgb]{0.56,0.35,0.01}{\textbf{\textit{#1}}}}
\newcommand{\OtherTok}[1]{\textcolor[rgb]{0.56,0.35,0.01}{#1}}
\newcommand{\FunctionTok}[1]{\textcolor[rgb]{0.00,0.00,0.00}{#1}}
\newcommand{\VariableTok}[1]{\textcolor[rgb]{0.00,0.00,0.00}{#1}}
\newcommand{\ControlFlowTok}[1]{\textcolor[rgb]{0.13,0.29,0.53}{\textbf{#1}}}
\newcommand{\OperatorTok}[1]{\textcolor[rgb]{0.81,0.36,0.00}{\textbf{#1}}}
\newcommand{\BuiltInTok}[1]{#1}
\newcommand{\ExtensionTok}[1]{#1}
\newcommand{\PreprocessorTok}[1]{\textcolor[rgb]{0.56,0.35,0.01}{\textit{#1}}}
\newcommand{\AttributeTok}[1]{\textcolor[rgb]{0.77,0.63,0.00}{#1}}
\newcommand{\RegionMarkerTok}[1]{#1}
\newcommand{\InformationTok}[1]{\textcolor[rgb]{0.56,0.35,0.01}{\textbf{\textit{#1}}}}
\newcommand{\WarningTok}[1]{\textcolor[rgb]{0.56,0.35,0.01}{\textbf{\textit{#1}}}}
\newcommand{\AlertTok}[1]{\textcolor[rgb]{0.94,0.16,0.16}{#1}}
\newcommand{\ErrorTok}[1]{\textcolor[rgb]{0.64,0.00,0.00}{\textbf{#1}}}
\newcommand{\NormalTok}[1]{#1}
\usepackage{graphicx,grffile}
\makeatletter
\def\maxwidth{\ifdim\Gin@nat@width>\linewidth\linewidth\else\Gin@nat@width\fi}
\def\maxheight{\ifdim\Gin@nat@height>\textheight\textheight\else\Gin@nat@height\fi}
\makeatother
% Scale images if necessary, so that they will not overflow the page
% margins by default, and it is still possible to overwrite the defaults
% using explicit options in \includegraphics[width, height, ...]{}
\setkeys{Gin}{width=\maxwidth,height=\maxheight,keepaspectratio}
\IfFileExists{parskip.sty}{%
\usepackage{parskip}
}{% else
\setlength{\parindent}{0pt}
\setlength{\parskip}{6pt plus 2pt minus 1pt}
}
\setlength{\emergencystretch}{3em}  % prevent overfull lines
\providecommand{\tightlist}{%
  \setlength{\itemsep}{0pt}\setlength{\parskip}{0pt}}
\setcounter{secnumdepth}{0}
% Redefines (sub)paragraphs to behave more like sections
\ifx\paragraph\undefined\else
\let\oldparagraph\paragraph
\renewcommand{\paragraph}[1]{\oldparagraph{#1}\mbox{}}
\fi
\ifx\subparagraph\undefined\else
\let\oldsubparagraph\subparagraph
\renewcommand{\subparagraph}[1]{\oldsubparagraph{#1}\mbox{}}
\fi

%%% Use protect on footnotes to avoid problems with footnotes in titles
\let\rmarkdownfootnote\footnote%
\def\footnote{\protect\rmarkdownfootnote}

%%% Change title format to be more compact
\usepackage{titling}

% Create subtitle command for use in maketitle
\newcommand{\subtitle}[1]{
  \posttitle{
    \begin{center}\large#1\end{center}
    }
}

\setlength{\droptitle}{-2em}

  \title{Relação entre votos, receitas, despesas e bens de candidatos}
    \pretitle{\vspace{\droptitle}\centering\huge}
  \posttitle{\par}
    \author{Lucas Aragão}
    \preauthor{\centering\large\emph}
  \postauthor{\par}
    \date{}
    \predate{}\postdate{}
  

\begin{document}
\maketitle

Explique os dados que você vai analisar, de onde obteve e que análise
pretende fazer. Filtre candidatos de apenas um cargo e um lugar, que
tenha pelo menos 90 candidatos. Por exemplo, deputado estadual na
Paraíba ou vereador em Campina Grande.

\begin{Shaded}
\begin{Highlighting}[]
\CommentTok{# Filtre apenas o cargo que você está interessado em analisar. Exemplo:}
\NormalTok{dados_candidatos <-}\StringTok{ }\NormalTok{dados_candidatos }\OperatorTok
\StringTok{  }\KeywordTok{filter}\NormalTok{(Cargo }\OperatorTok{==}\StringTok{ "DEPUTADO FEDERAL"}\NormalTok{)}

\NormalTok{dados_candidatos}
\end{Highlighting}
\end{Shaded}

\begin{verbatim}
## # A tibble: 96 x 11
##    Cargo `Sigla Partido` `Sequencial Can~ `Numero candida~ `Nome candidato`
##    <chr> <chr>                      <dbl>            <int> <chr>           
##  1 DEPU~ DEM                 150000000206             2539 BENEDITA SIQUEI~
##  2 DEPU~ DEM                 150000000207             2511 EFRAIM DE ARAÚJ~
##  3 DEPU~ PC do B             150000000205             6565 PERCIVAL HENRIQ~
##  4 DEPU~ PCO                 150000000545             2929 CAMILO SOBREIRA~
##  5 DEPU~ PDT                 150000000194             1212 DAMIÃO FELICIAN~
##  6 DEPU~ PDT                 150000000195             1234 PAULO ROBERTO A~
##  7 DEPU~ PHS                 150000000209             3131 ALVARO GAUDENCI~
##  8 DEPU~ PHS                 150000000222             3100 ANGELA PATRICIA~
##  9 DEPU~ PMDB                150000000171             1555 ANDRÉ AUGUSTO C~
## 10 DEPU~ PMDB                150000000172             1500 EVISNALDO CRUZ ~
## # ... with 86 more rows, and 6 more variables: Turno <int>,
## #   Situacao <chr>, `Total votos` <int>, `Total despesas` <dbl>, `Total
## #   receitas` <dbl>, `Total bens` <dbl>
\end{verbatim}

\section{Análise exploratória}\label{analise-exploratoria}

Inicialmente, faça uma análise exploratória de cada variável
individualmente.

\begin{Shaded}
\begin{Highlighting}[]
\NormalTok{dados_candidatos}\OperatorTok
\KeywordTok{top_n}\NormalTok{(}\DecValTok{1}\NormalTok{, }\StringTok{`}\DataTypeTok{Total votos}\StringTok{`}\NormalTok{)}
\end{Highlighting}
\end{Shaded}

\begin{verbatim}
## # A tibble: 1 x 11
##   Cargo `Sigla Partido` `Sequencial Can~ `Numero candida~ `Nome candidato`
##   <chr> <chr>                      <dbl>            <int> <chr>           
## 1 DEPU~ PSDB                150000000507             4510 PEDRO OLIVEIRA ~
## # ... with 6 more variables: Turno <int>, Situacao <chr>, `Total
## #   votos` <int>, `Total despesas` <dbl>, `Total receitas` <dbl>, `Total
## #   bens` <dbl>
\end{verbatim}

\begin{Shaded}
\begin{Highlighting}[]
\NormalTok{dados_candidatos}\OperatorTok
\KeywordTok{top_n}\NormalTok{(}\OperatorTok{-}\DecValTok{1}\NormalTok{, }\StringTok{`}\DataTypeTok{Total votos}\StringTok{`}\NormalTok{)}
\end{Highlighting}
\end{Shaded}

\begin{verbatim}
## # A tibble: 6 x 11
##   Cargo `Sigla Partido` `Sequencial Can~ `Numero candida~ `Nome candidato`
##   <chr> <chr>                      <dbl>            <int> <chr>           
## 1 DEPU~ PPL                 150000000187             5400 JONATHAS ARRUDA~
## 2 DEPU~ PPL                 150000000188             5444 JOSE HONORATO D~
## 3 DEPU~ PRTB                150000000190             2899 CARLOS ANTONIO ~
## 4 DEPU~ PRTB                150000000528             2820 GIVALDO CESAR S~
## 5 DEPU~ PT                  150000000521             1314 JALES JAVA DOS ~
## 6 DEPU~ PV                  150000000533             4344 ADAILTO BARROS ~
## # ... with 6 more variables: Turno <int>, Situacao <chr>, `Total
## #   votos` <int>, `Total despesas` <dbl>, `Total receitas` <dbl>, `Total
## #   bens` <dbl>
\end{verbatim}

\begin{Shaded}
\begin{Highlighting}[]
\NormalTok{dados_candidatos}\OperatorTok
\KeywordTok{top_n}\NormalTok{(}\DecValTok{1}\NormalTok{, }\StringTok{`}\DataTypeTok{Total despesas}\StringTok{`}\NormalTok{)}
\end{Highlighting}
\end{Shaded}

\begin{verbatim}
## # A tibble: 1 x 11
##   Cargo `Sigla Partido` `Sequencial Can~ `Numero candida~ `Nome candidato`
##   <chr> <chr>                      <dbl>            <int> <chr>           
## 1 DEPU~ PR                  150000000492             2222 JOSÉ WELLINGTON~
## # ... with 6 more variables: Turno <int>, Situacao <chr>, `Total
## #   votos` <int>, `Total despesas` <dbl>, `Total receitas` <dbl>, `Total
## #   bens` <dbl>
\end{verbatim}

\begin{Shaded}
\begin{Highlighting}[]
\NormalTok{dados_candidatos}\OperatorTok
\KeywordTok{top_n}\NormalTok{(}\OperatorTok{-}\DecValTok{1}\NormalTok{, }\StringTok{`}\DataTypeTok{Total despesas}\StringTok{`}\NormalTok{)}
\end{Highlighting}
\end{Shaded}

\begin{verbatim}
## # A tibble: 1 x 11
##   Cargo `Sigla Partido` `Sequencial Can~ `Numero candida~ `Nome candidato`
##   <chr> <chr>                      <dbl>            <int> <chr>           
## 1 DEPU~ PTC                 150000000008             3699 JOSÉ ALBERTO MA~
## # ... with 6 more variables: Turno <int>, Situacao <chr>, `Total
## #   votos` <int>, `Total despesas` <dbl>, `Total receitas` <dbl>, `Total
## #   bens` <dbl>
\end{verbatim}

\begin{Shaded}
\begin{Highlighting}[]
\KeywordTok{ggplot}\NormalTok{(dados_candidatos, }\KeywordTok{aes}\NormalTok{(}\StringTok{`}\DataTypeTok{Total despesas}\StringTok{`}\NormalTok{)) }\OperatorTok{+}
\KeywordTok{geom_histogram}\NormalTok{()}
\end{Highlighting}
\end{Shaded}

\begin{verbatim}
## `stat_bin()` using `bins = 30`. Pick better value with `binwidth`.
\end{verbatim}

\begin{verbatim}
## Warning: Removed 18 rows containing non-finite values (stat_bin).
\end{verbatim}

\includegraphics{relatorio_files/figure-latex/unnamed-chunk-5-1.pdf}

\begin{Shaded}
\begin{Highlighting}[]
\KeywordTok{ggplot}\NormalTok{(dados_candidatos, }\KeywordTok{aes}\NormalTok{(}\StringTok{`}\DataTypeTok{Total bens}\StringTok{`}\NormalTok{)) }\OperatorTok{+}
\KeywordTok{geom_histogram}\NormalTok{()}
\end{Highlighting}
\end{Shaded}

\begin{verbatim}
## `stat_bin()` using `bins = 30`. Pick better value with `binwidth`.
\end{verbatim}

\begin{verbatim}
## Warning: Removed 33 rows containing non-finite values (stat_bin).
\end{verbatim}

\includegraphics{relatorio_files/figure-latex/unnamed-chunk-6-1.pdf}

\begin{Shaded}
\begin{Highlighting}[]
\NormalTok{dados_candidatos }\OperatorTok
\StringTok{  }\KeywordTok{filter}\NormalTok{(}\StringTok{`}\DataTypeTok{Sigla Partido}\StringTok{`} \OperatorTok{==}\StringTok{ "PT"}\NormalTok{) }
\end{Highlighting}
\end{Shaded}

\begin{verbatim}
## # A tibble: 7 x 11
##   Cargo `Sigla Partido` `Sequencial Can~ `Numero candida~ `Nome candidato`
##   <chr> <chr>                      <dbl>            <int> <chr>           
## 1 DEPU~ PT                  150000000198             1323 JOSÉ LEONCIO DA~
## 2 DEPU~ PT                  150000000199             1345 LUIZ ALBUQUERQU~
## 3 DEPU~ PT                  150000000200             1301 MARIA DAS GRAÇA~
## 4 DEPU~ PT                  150000000201             1333 ODON BEZERRA CA~
## 5 DEPU~ PT                  150000000202             1311 RICARDO CARDOSO~
## 6 DEPU~ PT                  150000000204             1300 UBIRATAN PEREIR~
## 7 DEPU~ PT                  150000000521             1314 JALES JAVA DOS ~
## # ... with 6 more variables: Turno <int>, Situacao <chr>, `Total
## #   votos` <int>, `Total despesas` <dbl>, `Total receitas` <dbl>, `Total
## #   bens` <dbl>
\end{verbatim}

\begin{Shaded}
\begin{Highlighting}[]
\NormalTok{dados_candidatos}\OperatorTok
\KeywordTok{filter}\NormalTok{(}\StringTok{`}\DataTypeTok{Sigla Partido}\StringTok{`} \OperatorTok{==}\StringTok{ "PSDB"}\NormalTok{)}
\end{Highlighting}
\end{Shaded}

\begin{verbatim}
## # A tibble: 5 x 11
##   Cargo `Sigla Partido` `Sequencial Can~ `Numero candida~ `Nome candidato`
##   <chr> <chr>                      <dbl>            <int> <chr>           
## 1 DEPU~ PSDB                150000000502             4555 EDIRCE DE OLIVE~
## 2 DEPU~ PSDB                150000000503             4545 EMERSON FERNAND~
## 3 DEPU~ PSDB                150000000504             4522 IRAÊ HEUSI DE L~
## 4 DEPU~ PSDB                150000000505             4515 JOÃO BATISTA FR~
## 5 DEPU~ PSDB                150000000507             4510 PEDRO OLIVEIRA ~
## # ... with 6 more variables: Turno <int>, Situacao <chr>, `Total
## #   votos` <int>, `Total despesas` <dbl>, `Total receitas` <dbl>, `Total
## #   bens` <dbl>
\end{verbatim}

\begin{Shaded}
\begin{Highlighting}[]
\NormalTok{dados_candidatos}\OperatorTok
\KeywordTok{filter}\NormalTok{(}\StringTok{`}\DataTypeTok{Sigla Partido}\StringTok{`} \OperatorTok{==}\StringTok{ "DEM"}\NormalTok{)}
\end{Highlighting}
\end{Shaded}

\begin{verbatim}
## # A tibble: 2 x 11
##   Cargo `Sigla Partido` `Sequencial Can~ `Numero candida~ `Nome candidato`
##   <chr> <chr>                      <dbl>            <int> <chr>           
## 1 DEPU~ DEM                 150000000206             2539 BENEDITA SIQUEI~
## 2 DEPU~ DEM                 150000000207             2511 EFRAIM DE ARAÚJ~
## # ... with 6 more variables: Turno <int>, Situacao <chr>, `Total
## #   votos` <int>, `Total despesas` <dbl>, `Total receitas` <dbl>, `Total
## #   bens` <dbl>
\end{verbatim}

\begin{Shaded}
\begin{Highlighting}[]
\NormalTok{dados_candidatos }\OperatorTok
\StringTok{  }\KeywordTok{filter}\NormalTok{(}\StringTok{`}\DataTypeTok{Sigla Partido}\StringTok{`} \OperatorTok{==}\StringTok{ "PMDB"}\NormalTok{)}
\end{Highlighting}
\end{Shaded}

\begin{verbatim}
## # A tibble: 8 x 11
##   Cargo `Sigla Partido` `Sequencial Can~ `Numero candida~ `Nome candidato`
##   <chr> <chr>                      <dbl>            <int> <chr>           
## 1 DEPU~ PMDB                150000000171             1555 ANDRÉ AUGUSTO C~
## 2 DEPU~ PMDB                150000000172             1500 EVISNALDO CRUZ ~
## 3 DEPU~ PMDB                150000000173             1522 HUGO MOTTA WAND~
## 4 DEPU~ PMDB                150000000174             1533 JAQUELINE BARBO~
## 5 DEPU~ PMDB                150000000175             1515 MANOEL ALVES DA~
## 6 DEPU~ PMDB                150000000176             1511 MARIA AUXILIADO~
## 7 DEPU~ PMDB                150000000177             1525 MARIA CARMÉN SI~
## 8 DEPU~ PMDB                150000000178             1510 VENEZIANO VITAL~
## # ... with 6 more variables: Turno <int>, Situacao <chr>, `Total
## #   votos` <int>, `Total despesas` <dbl>, `Total receitas` <dbl>, `Total
## #   bens` <dbl>
\end{verbatim}

\section{Análise das relações entre
variáveis}\label{analise-das-relacoes-entre-variaveis}

Faça uma análise das relações entre as variáveis em questão. Você pode
começar com uma análise gráfica das relações entre variáveis, para
depois partir para análise de correlação e regressão.

lm == linear model

\begin{Shaded}
\begin{Highlighting}[]
\KeywordTok{ggplot}\NormalTok{(dados_candidatos, }\KeywordTok{aes}\NormalTok{(}\DataTypeTok{x =} \StringTok{`}\DataTypeTok{Total votos}\StringTok{`}\NormalTok{, }\DataTypeTok{y =} \StringTok{`}\DataTypeTok{Total despesas}\StringTok{`}\NormalTok{)) }\OperatorTok{+}\StringTok{ }
\StringTok{  }\KeywordTok{geom_point}\NormalTok{(}\DataTypeTok{alpha =} \FloatTok{0.4}\NormalTok{) }\OperatorTok{+}\StringTok{ }\KeywordTok{geom_smooth}\NormalTok{(}\DataTypeTok{method =} \StringTok{"lm"}\NormalTok{, }\DataTypeTok{se =} \OtherTok{FALSE}\NormalTok{)}
\end{Highlighting}
\end{Shaded}

\begin{verbatim}
## Warning: Removed 18 rows containing non-finite values (stat_smooth).
\end{verbatim}

\begin{verbatim}
## Warning: Removed 18 rows containing missing values (geom_point).
\end{verbatim}

\includegraphics{relatorio_files/figure-latex/unnamed-chunk-11-1.pdf}

\begin{Shaded}
\begin{Highlighting}[]
\KeywordTok{ggplot}\NormalTok{(dados_candidatos, }\KeywordTok{aes}\NormalTok{(}\DataTypeTok{x =} \StringTok{`}\DataTypeTok{Total votos}\StringTok{`}\NormalTok{, }\DataTypeTok{y =} \StringTok{`}\DataTypeTok{Total bens}\StringTok{`}\NormalTok{)) }\OperatorTok{+}\StringTok{ }
\StringTok{  }\KeywordTok{geom_point}\NormalTok{(}\DataTypeTok{alpha =} \FloatTok{0.4}\NormalTok{) }\OperatorTok{+}\StringTok{ }\KeywordTok{geom_smooth}\NormalTok{(}\DataTypeTok{method =} \StringTok{"lm"}\NormalTok{, }\DataTypeTok{se =} \OtherTok{FALSE}\NormalTok{)}
\end{Highlighting}
\end{Shaded}

\begin{verbatim}
## Warning: Removed 33 rows containing non-finite values (stat_smooth).
\end{verbatim}

\begin{verbatim}
## Warning: Removed 33 rows containing missing values (geom_point).
\end{verbatim}

\includegraphics{relatorio_files/figure-latex/unnamed-chunk-12-1.pdf}

\begin{Shaded}
\begin{Highlighting}[]
\NormalTok{modelo <-}\StringTok{ }\KeywordTok{lm}\NormalTok{(}\StringTok{`}\DataTypeTok{Total votos}\StringTok{`} \OperatorTok{~}\StringTok{ `}\DataTypeTok{Total despesas}\StringTok{`}\NormalTok{, }\DataTypeTok{data =}\NormalTok{ dados_candidatos)}
  
  \KeywordTok{summary}\NormalTok{(modelo)}
\end{Highlighting}
\end{Shaded}

\begin{verbatim}
## 
## Call:
## lm(formula = `Total votos` ~ `Total despesas`, data = dados_candidatos)
## 
## Residuals:
##    Min     1Q Median     3Q    Max 
## -99668  -6407  -5890     50  94084 
## 
## Coefficients:
##                   Estimate Std. Error t value Pr(>|t|)    
## (Intercept)      6,504e+03  2,839e+03   2,291   0,0247 *  
## `Total despesas` 8,899e-02  6,408e-03  13,888   <2e-16 ***
## ---
## Signif. codes:  0 '***' 0,001 '**' 0,01 '*' 0,05 '.' 0,1 ' ' 1
## 
## Residual standard error: 22870 on 76 degrees of freedom
##   (18 observations deleted due to missingness)
## Multiple R-squared:  0,7173, Adjusted R-squared:  0,7136 
## F-statistic: 192,9 on 1 and 76 DF,  p-value: < 2,2e-16
\end{verbatim}

\begin{Shaded}
\begin{Highlighting}[]
  \KeywordTok{confint}\NormalTok{(modelo)}
\end{Highlighting}
\end{Shaded}

\begin{verbatim}
##                         2,5 %       97,5 %
## (Intercept)      850,28966272 12158,289830
## `Total despesas`   0,07622912     0,101754
\end{verbatim}

\begin{Shaded}
\begin{Highlighting}[]
  \KeywordTok{tidy}\NormalTok{(modelo, }\DataTypeTok{confint=} \OtherTok{TRUE}\NormalTok{)}
\end{Highlighting}
\end{Shaded}

\begin{verbatim}
## # A tibble: 2 x 5
##   term              estimate  std.error statistic  p.value
##   <chr>                <dbl>      <dbl>     <dbl>    <dbl>
## 1 (Intercept)      6504.     2839.           2.29 2.47e- 2
## 2 `Total despesas`    0.0890    0.00641     13.9  1.51e-22
\end{verbatim}

\begin{Shaded}
\begin{Highlighting}[]
  \KeywordTok{glance}\NormalTok{(modelo)}
\end{Highlighting}
\end{Shaded}

\begin{verbatim}
## # A tibble: 1 x 11
##   r.squared adj.r.squared  sigma statistic  p.value    df logLik   AIC
## *     <dbl>         <dbl>  <dbl>     <dbl>    <dbl> <int>  <dbl> <dbl>
## 1     0.717         0.714 22871.      193. 1.51e-22     2  -893. 1791.
## # ... with 3 more variables: BIC <dbl>, deviance <dbl>, df.residual <int>
\end{verbatim}

\begin{Shaded}
\begin{Highlighting}[]
\NormalTok{  dados_candidatos }\OperatorTok
\StringTok{    }\KeywordTok{add_predictions}\NormalTok{(}\DataTypeTok{model =}\NormalTok{ modelo) }\OperatorTok
\StringTok{    }\KeywordTok{ggplot}\NormalTok{(}\DataTypeTok{mapping =} \KeywordTok{aes}\NormalTok{(}\DataTypeTok{x =} \StringTok{`}\DataTypeTok{Total despesas}\StringTok{`}\NormalTok{, }\DataTypeTok{y=} \StringTok{`}\DataTypeTok{Total votos}\StringTok{`}\NormalTok{)) }\OperatorTok{+}
\StringTok{    }\KeywordTok{geom_point}\NormalTok{(}\DataTypeTok{alpha =} \FloatTok{0.5}\NormalTok{, }\DataTypeTok{size =}\NormalTok{ .}\DecValTok{8}\NormalTok{) }\OperatorTok{+}\StringTok{ }
\StringTok{    }\KeywordTok{geom_line}\NormalTok{(}\KeywordTok{aes}\NormalTok{(}\DataTypeTok{y =}\NormalTok{ pred), }\DataTypeTok{colour =} \StringTok{"blue"}\NormalTok{)}
\end{Highlighting}
\end{Shaded}

\begin{verbatim}
## Warning: Removed 18 rows containing missing values (geom_point).
\end{verbatim}

\begin{verbatim}
## Warning: Removed 18 rows containing missing values (geom_path).
\end{verbatim}

\includegraphics{relatorio_files/figure-latex/unnamed-chunk-13-1.pdf}

\begin{Shaded}
\begin{Highlighting}[]
\NormalTok{modelo <-}\StringTok{ }\KeywordTok{lm}\NormalTok{(}\StringTok{`}\DataTypeTok{Total votos}\StringTok{`} \OperatorTok{~}\StringTok{ `}\DataTypeTok{Total receitas}\StringTok{`}\NormalTok{, }\DataTypeTok{data =}\NormalTok{ dados_candidatos)}
  
  \KeywordTok{summary}\NormalTok{(modelo)}
\end{Highlighting}
\end{Shaded}

\begin{verbatim}
## 
## Call:
## lm(formula = `Total votos` ~ `Total receitas`, data = dados_candidatos)
## 
## Residuals:
##     Min      1Q  Median      3Q     Max 
## -106377   -6637   -5854    -649   92017 
## 
## Coefficients:
##                   Estimate Std. Error t value Pr(>|t|)    
## (Intercept)      6,740e+03  2,842e+03   2,371   0,0203 *  
## `Total receitas` 9,102e-02  6,584e-03  13,823   <2e-16 ***
## ---
## Signif. codes:  0 '***' 0,001 '**' 0,01 '*' 0,05 '.' 0,1 ' ' 1
## 
## Residual standard error: 22950 on 76 degrees of freedom
##   (18 observations deleted due to missingness)
## Multiple R-squared:  0,7154, Adjusted R-squared:  0,7117 
## F-statistic: 191,1 on 1 and 76 DF,  p-value: < 2,2e-16
\end{verbatim}

\begin{Shaded}
\begin{Highlighting}[]
  \KeywordTok{confint}\NormalTok{(modelo)}
\end{Highlighting}
\end{Shaded}

\begin{verbatim}
##                         2,5 %       97,5 %
## (Intercept)      1,079211e+03 1,240003e+04
## `Total receitas` 7,790291e-02 1,041311e-01
\end{verbatim}

\begin{Shaded}
\begin{Highlighting}[]
  \KeywordTok{tidy}\NormalTok{(modelo, }\DataTypeTok{confint=} \OtherTok{TRUE}\NormalTok{)}
\end{Highlighting}
\end{Shaded}

\begin{verbatim}
## # A tibble: 2 x 5
##   term              estimate  std.error statistic  p.value
##   <chr>                <dbl>      <dbl>     <dbl>    <dbl>
## 1 (Intercept)      6740.     2842.           2.37 2.03e- 2
## 2 `Total receitas`    0.0910    0.00658     13.8  1.95e-22
\end{verbatim}

\begin{Shaded}
\begin{Highlighting}[]
  \KeywordTok{glance}\NormalTok{(modelo)}
\end{Highlighting}
\end{Shaded}

\begin{verbatim}
## # A tibble: 1 x 11
##   r.squared adj.r.squared  sigma statistic  p.value    df logLik   AIC
## *     <dbl>         <dbl>  <dbl>     <dbl>    <dbl> <int>  <dbl> <dbl>
## 1     0.715         0.712 22948.      191. 1.95e-22     2  -893. 1792.
## # ... with 3 more variables: BIC <dbl>, deviance <dbl>, df.residual <int>
\end{verbatim}

\begin{Shaded}
\begin{Highlighting}[]
\NormalTok{  dados_candidatos }\OperatorTok
\StringTok{    }\KeywordTok{add_predictions}\NormalTok{(}\DataTypeTok{model =}\NormalTok{ modelo) }\OperatorTok
\StringTok{    }\KeywordTok{ggplot}\NormalTok{(}\DataTypeTok{mapping =} \KeywordTok{aes}\NormalTok{(}\DataTypeTok{x =} \StringTok{`}\DataTypeTok{Total receitas}\StringTok{`}\NormalTok{, }\DataTypeTok{y=} \StringTok{`}\DataTypeTok{Total votos}\StringTok{`}\NormalTok{)) }\OperatorTok{+}
\StringTok{    }\KeywordTok{geom_point}\NormalTok{(}\DataTypeTok{alpha =} \FloatTok{0.5}\NormalTok{, }\DataTypeTok{size =}\NormalTok{ .}\DecValTok{8}\NormalTok{) }\OperatorTok{+}\StringTok{ }
\StringTok{    }\KeywordTok{geom_line}\NormalTok{(}\KeywordTok{aes}\NormalTok{(}\DataTypeTok{y =}\NormalTok{ pred), }\DataTypeTok{colour =} \StringTok{"blue"}\NormalTok{)}
\end{Highlighting}
\end{Shaded}

\begin{verbatim}
## Warning: Removed 18 rows containing missing values (geom_point).
\end{verbatim}

\begin{verbatim}
## Warning: Removed 18 rows containing missing values (geom_path).
\end{verbatim}

\includegraphics{relatorio_files/figure-latex/unnamed-chunk-14-1.pdf}

\begin{Shaded}
\begin{Highlighting}[]
\NormalTok{modelo <-}\StringTok{ }\KeywordTok{lm}\NormalTok{(}\StringTok{`}\DataTypeTok{Total votos}\StringTok{`} \OperatorTok{~}\StringTok{ `}\DataTypeTok{Total bens}\StringTok{`}\NormalTok{, }\DataTypeTok{data =}\NormalTok{ dados_candidatos)}
  
  \KeywordTok{summary}\NormalTok{(modelo)}
\end{Highlighting}
\end{Shaded}

\begin{verbatim}
## 
## Call:
## lm(formula = `Total votos` ~ `Total bens`, data = dados_candidatos)
## 
## Residuals:
##    Min     1Q Median     3Q    Max 
## -48013 -17787 -14106   2104 163216 
## 
## Coefficients:
##               Estimate Std. Error t value Pr(>|t|)    
## (Intercept)  1,376e+04  5,677e+03   2,423   0,0184 *  
## `Total bens` 2,317e-02  4,791e-03   4,836 9,35e-06 ***
## ---
## Signif. codes:  0 '***' 0,001 '**' 0,01 '*' 0,05 '.' 0,1 ' ' 1
## 
## Residual standard error: 39730 on 61 degrees of freedom
##   (33 observations deleted due to missingness)
## Multiple R-squared:  0,2771, Adjusted R-squared:  0,2653 
## F-statistic: 23,39 on 1 and 61 DF,  p-value: 9,347e-06
\end{verbatim}

\begin{Shaded}
\begin{Highlighting}[]
  \KeywordTok{confint}\NormalTok{(modelo)}
\end{Highlighting}
\end{Shaded}

\begin{verbatim}
##                     2,5 %       97,5 %
## (Intercept)  2,406164e+03 2,511162e+04
## `Total bens` 1,358833e-02 3,274685e-02
\end{verbatim}

\begin{Shaded}
\begin{Highlighting}[]
  \KeywordTok{tidy}\NormalTok{(modelo, }\DataTypeTok{confint=} \OtherTok{TRUE}\NormalTok{)}
\end{Highlighting}
\end{Shaded}

\begin{verbatim}
## # A tibble: 2 x 5
##   term           estimate  std.error statistic    p.value
##   <chr>             <dbl>      <dbl>     <dbl>      <dbl>
## 1 (Intercept)  13759.     5677.           2.42 0.0184    
## 2 `Total bens`     0.0232    0.00479      4.84 0.00000935
\end{verbatim}

\begin{Shaded}
\begin{Highlighting}[]
  \KeywordTok{glance}\NormalTok{(modelo)}
\end{Highlighting}
\end{Shaded}

\begin{verbatim}
## # A tibble: 1 x 11
##   r.squared adj.r.squared  sigma statistic p.value    df logLik   AIC   BIC
## *     <dbl>         <dbl>  <dbl>     <dbl>   <dbl> <int>  <dbl> <dbl> <dbl>
## 1     0.277         0.265 39732.      23.4 9.35e-6     2  -756. 1517. 1524.
## # ... with 2 more variables: deviance <dbl>, df.residual <int>
\end{verbatim}

\begin{Shaded}
\begin{Highlighting}[]
\NormalTok{  dados_candidatos }\OperatorTok
\StringTok{    }\KeywordTok{add_predictions}\NormalTok{(}\DataTypeTok{model =}\NormalTok{ modelo) }\OperatorTok
\StringTok{    }\KeywordTok{ggplot}\NormalTok{(}\DataTypeTok{mapping =} \KeywordTok{aes}\NormalTok{(}\DataTypeTok{x =} \StringTok{`}\DataTypeTok{Total bens}\StringTok{`}\NormalTok{, }\DataTypeTok{y=} \StringTok{`}\DataTypeTok{Total votos}\StringTok{`}\NormalTok{)) }\OperatorTok{+}
\StringTok{    }\KeywordTok{geom_point}\NormalTok{(}\DataTypeTok{alpha =} \FloatTok{0.5}\NormalTok{, }\DataTypeTok{size =}\NormalTok{ .}\DecValTok{8}\NormalTok{) }\OperatorTok{+}\StringTok{ }
\StringTok{    }\KeywordTok{geom_line}\NormalTok{(}\KeywordTok{aes}\NormalTok{(}\DataTypeTok{y =}\NormalTok{ pred), }\DataTypeTok{colour =} \StringTok{"blue"}\NormalTok{)}
\end{Highlighting}
\end{Shaded}

\begin{verbatim}
## Warning: Removed 33 rows containing missing values (geom_point).
\end{verbatim}

\begin{verbatim}
## Warning: Removed 33 rows containing missing values (geom_path).
\end{verbatim}

\includegraphics{relatorio_files/figure-latex/unnamed-chunk-15-1.pdf}

\begin{Shaded}
\begin{Highlighting}[]
\CommentTok{# Filtre apenas o cargo que você está interessado em analisar. Exemplo:}
\NormalTok{dados_candidatos2 <-}\StringTok{ }\NormalTok{dados_candidatos2 }\OperatorTok
\StringTok{  }\KeywordTok{filter}\NormalTok{(Cargo }\OperatorTok{==}\StringTok{ "DEPUTADO FEDERAL"}\NormalTok{)}

\NormalTok{dados_candidatos2}
\end{Highlighting}
\end{Shaded}

\begin{verbatim}
## # A tibble: 155 x 11
##    Cargo `Sigla Partido` `Sequencial Can~ `Numero candida~ `Nome candidato`
##    <chr> <chr>                      <dbl>            <int> <chr>           
##  1 DEPU~ DEM                 170000000076             2525 JOSÉ MENDONÇA B~
##  2 DEPU~ PC do B             170000000030             6540 CARLOS EDUARDO ~
##  3 DEPU~ PC do B             170000000052             6510 LUCIANA BARBOSA~
##  4 DEPU~ PCB                 170000000013             2121 GERLANE SIMÕES ~
##  5 DEPU~ PCB                 170000000017             2112 EDVALMIR SOARES~
##  6 DEPU~ PCB                 170000000018             2111 JERONIMO NEIVA ~
##  7 DEPU~ PCB                 170000000019             2122 ANTONIO DE CAST~
##  8 DEPU~ PCB                 170000000023             2133 SILVIA ALBUQUER~
##  9 DEPU~ PDT                 170000000604             1240 MARILEIDE ROSEN~
## 10 DEPU~ PDT                 170000000615             1212 ISABELLA MENEZE~
## # ... with 145 more rows, and 6 more variables: Turno <int>,
## #   Situacao <chr>, `Total votos` <int>, `Total despesas` <dbl>, `Total
## #   receitas` <dbl>, `Total bens` <dbl>
\end{verbatim}

\begin{Shaded}
\begin{Highlighting}[]
\NormalTok{dados_candidatos2}\OperatorTok
\StringTok{  }\KeywordTok{top_n}\NormalTok{(}\DecValTok{1}\NormalTok{, }\StringTok{`}\DataTypeTok{Total votos}\StringTok{`}\NormalTok{)}
\end{Highlighting}
\end{Shaded}

\begin{verbatim}
## # A tibble: 1 x 11
##   Cargo `Sigla Partido` `Sequencial Can~ `Numero candida~ `Nome candidato`
##   <chr> <chr>                      <dbl>            <int> <chr>           
## 1 DEPU~ PP                  170000000046             1111 EDUARDO HENRIQU~
## # ... with 6 more variables: Turno <int>, Situacao <chr>, `Total
## #   votos` <int>, `Total despesas` <dbl>, `Total receitas` <dbl>, `Total
## #   bens` <dbl>
\end{verbatim}

\begin{Shaded}
\begin{Highlighting}[]
\NormalTok{dados_candidatos2 }\OperatorTok
\StringTok{  }\KeywordTok{top_n}\NormalTok{(}\DecValTok{1}\NormalTok{, }\StringTok{`}\DataTypeTok{Total despesas}\StringTok{`}\NormalTok{)}
\end{Highlighting}
\end{Shaded}

\begin{verbatim}
## # A tibble: 0 x 11
## # ... with 11 variables: Cargo <chr>, `Sigla Partido` <chr>, `Sequencial
## #   Candidato` <dbl>, `Numero candidato` <int>, `Nome candidato` <chr>,
## #   Turno <int>, Situacao <chr>, `Total votos` <int>, `Total
## #   despesas` <dbl>, `Total receitas` <dbl>, `Total bens` <dbl>
\end{verbatim}

\begin{Shaded}
\begin{Highlighting}[]
\NormalTok{dados_candidatos2 }\OperatorTok
\StringTok{  }\KeywordTok{top_n}\NormalTok{(}\OperatorTok{-}\DecValTok{1}\NormalTok{, }\StringTok{`}\DataTypeTok{Total despesas}\StringTok{`}\NormalTok{)}
\end{Highlighting}
\end{Shaded}

\begin{verbatim}
## # A tibble: 0 x 11
## # ... with 11 variables: Cargo <chr>, `Sigla Partido` <chr>, `Sequencial
## #   Candidato` <dbl>, `Numero candidato` <int>, `Nome candidato` <chr>,
## #   Turno <int>, Situacao <chr>, `Total votos` <int>, `Total
## #   despesas` <dbl>, `Total receitas` <dbl>, `Total bens` <dbl>
\end{verbatim}

\section{Análise de regressão}\label{analise-de-regressao}

Analise a relação entre a quantidade de votos que um candidato recebe
com outras variáveis. Você pode começar com regressões simples,
considerando apenas uma variável por vez, e em seguida fazer regressões
múltiplas.

Descreva bem as suas conclusões, de forma clara para um público geral.

\section{Exemplos de perguntas a
responder:}\label{exemplos-de-perguntas-a-responder}

\begin{itemize}
\tightlist
\item
  Quanto um candidato deve gastar no mínimo (despesa) para ter mais
  chance de ser eleito?
\item
  Quanto um candidato deve arrecadar no mínimo (receita) para ter mais
  chance de ser eleito?
\item
  Quanto de bens um candidato deve ter no mínimo para ter mais chance de
  ser eleito?
\item
  Existe alguma combinação de variáveis que explica bem a quantidade de
  dados que um candidato recebe?
\item
  Considere um candidato fictício que recebeu 100 mil reais de receita
  na campanha. Qual a previsão de votos que ele deve receber? É provável
  que ele seja eleito?
\item
  Considere um candidato fictício que gastou 100 mil reais de despesa na
  campanha e possui 1 milhão de reais em bens. Qual a previsão de votos
  que ele deve receber? É provável que ele seja eleito?
\end{itemize}


\end{document}
